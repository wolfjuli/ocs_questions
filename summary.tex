\documentclass{article}
\usepackage[utf8]{inputenc}
\usepackage[english]{babel}
\usepackage{graphicx}
\usepackage{amsmath}
\usepackage{amssymb}
\usepackage{commath}

\graphicspath{{figures/}}

\newcommand{\RR}{\mathbb{R}}
\newcommand{\NN}{\mathbb{N}}
\newcommand{\ZZ}{\mathbb{Z}}
\newcommand{\T}[1]{#1^{\top}}

\title{Summary of Slides}
\author{Philipp Gabler}

\begin{document}
\maketitle

%%%%%%%%%%%%%%%%%%%%%%%%%%%%%%%%%%%%%%%%%%%%%%%%%%%%%%%%%%%%%%%%%%
\section{Introduction}

\paragraph{General Form.} A general minimization problem has the form
\begin{equation*}
  \min_{x} f(x) \quad \text{s.t. } x \in X,
\end{equation*}
for a \emph{constraint set} \(X \subseteq \RR^n\) (often given by some \emph{constraint functions}
and an \emph{objective function} \(f: X \to \RR\).  We want to find an optimal value or
\emph{minimizer} \(x^* \in X\) such that
\begin{equation*}
  f(x^*) \leq f(x), \quad \forall x \in X.
\end{equation*}

\paragraph{Types of Optimization Problems.}
\begin{enumerate}
\item
  \begin{enumerate}
  \item Discrete: \(X\) is a discrete set, also called \emph{interger programming}.
  \item Continuous: \(X\) is continuous (ie. uncountable)
  \end{enumerate}
\item
  \begin{enumerate}
  \item Linear: Objective functions and constraints are all linear:
    \begin{equation*}
      \min_x \T{c} x, \quad\text{s.t. } Ax \leq b,\; x \geq 0.
    \end{equation*}
    Constraints describe a polyhedron.  Efficiently solvable.
  \item Quadratic: Objective function is quadratic, constraints linear:
    \begin{equation*}
      \min_x \frac{1}{2}\T{x} Q x + \T{c} x, \quad\text{s.t. } Ax \leq b,\; Ex = d.
    \end{equation*}
    If \(Q\) is positive semidefinite, the objective is convex and the problem is polynomially
    solvable.
  \item Nonlinear: no further constraints.
  \end{enumerate}
\item
  \begin{enumerate}
  \item Unconstrained: Optimal solution searched in full \(\RR^n\). Easier to characterize, and
    usually to solve.
  \item Constrained: Optimal solution in an admissible region, usually more difficult to
    setup/characterize.
  \end{enumerate}
\end{enumerate}

\paragraph{Convexity.}
A set \(X\) is convex, if for all \(x, y \in X\) and \(\alpha \in [0,1]\):
\begin{equation*}
  \alpha x + (1 - \alpha) y \in X.
\end{equation*}
This means that \(X\) contains all convex combinations of points from it.

\paragraph{Convex Functions.}
If \(X\) is a convex set, then \(f: \RR \to \RR\) is called convex if for all \(x, y \in X\) and
\(\alpha \in [0,1]\):
\begin{equation*}
  f(\alpha x + (1 - \alpha) y) \leq \alpha f(x) + (1 - \alpha) f(y).
\end{equation*}


\end{document}
