% graphics
\usepackage{graphicx}
\usepackage{float}
\graphicspath{{figures/}}

% todo notes
\usepackage{todonotes}
\let\oldtodo\todo
\renewcommand{\todo}[1]{\oldtodo[noline]{#1}}

% math stuff
\usepackage{amsmath}
\usepackage{amssymb}
\usepackage{amsthm}
\usepackage{commath}
\newcommand{\RR}{\mathbb{R}}
\newcommand{\NN}{\mathbb{N}}
\newcommand{\ZZ}{\mathbb{Z}}
\newcommand{\T}[1]{#1^{\text{T}}}
\newcommand{\kth}[2][k]{#2^{(#1)}}
\newcommand{\Lip}[3]{\mathcal{F}_{#1}^{#2, #3}}
\newcommand{\Strong}[4]{\mathcal{S}_{#1, #2}^{#3, #4}}
\newcommand{\strong}[2]{\mathcal{S}_{#1}^{#2}}
\newcommand{\Lipl}{\Lip{L}{1}{1}}
\newcommand{\strongm}{\strong{\mu}{1}}
\newcommand{\Strongml}{\Strong{L}{\mu}{1}{1}}
\newcommand{\iprod}[2]{\left\langle #1, #2 \right\rangle}

\DeclareMathOperator{\diag}{diag}
\DeclareMathOperator*{\argmin}{arg min}
\DeclareMathOperator{\Span}{span}
\DeclareMathOperator{\proj}{proj}
\DeclareMathOperator{\card}{card}
\DeclareMathOperator{\tr}{tr}
\DeclareMathOperator{\rank}{rank}

\hyphenation{Lip-schitz}

% layout/sectioning
\newcounter{summary}[section]
\newcommand{\summary}[1]{%
  \paragraph{\arabic{section}.\arabic{summary}\texorpdfstring{\enspace}{ } #1.}\refstepcounter{summary}}
\newcommand{\optionalsummary}[1]{%
  \paragraph{\arabic{section}.\arabic{summary}*\texorpdfstring{\enspace}{ } #1.}\refstepcounter{summary}}
\renewcommand{\thesummary}{\thesection .\arabic{summary}}

% todo: fix pdf bookmarks/make this a proper sectioning command
% \pdfbookmark[3]{#1}{subsec:\thesummary}%
% https://tex.stackexchange.com/questions/3881/formatting-a-paragraph-to-look-like-a-section

% fix (= remove) space below theorem environment, since we abuse it for question formatting
\newtheoremstyle{slplain}% name
  {.5\baselineskip\@plus.2\baselineskip\@minus.2\baselineskip}% Space above
  {0pt}% Space below
  {\itshape}% Body font
  {}%Indent amount (empty = no indent, \parindent = para indent)
  {\bfseries}%  Thm head font
  {.}%       Punctuation after thm head
  { }%      Space after thm head: " " = normal interword space;
        %       \newline = linebreak
  {}%       Thm head spec
\theoremstyle{slplain}
\newtheorem{question}{Question}

\usepackage{parskip}
